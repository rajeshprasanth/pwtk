\documentclass[12pt]{article}
\usepackage{amsmath}
\usepackage{geometry}
\geometry{a4paper, margin=1in}
\usepackage{hyperref}
\usepackage{graphicx}

\title{POSCAR Strain Application Script Documentation}
\author{Rajesh Prashanth A \\ \texttt{rajeshprasanth@rediffmail.com}}
\date{Version 1.0.0, Thu Apr 4, 2024}

\begin{document}
\maketitle

\section{Overview}
This Python script applies controlled strain to a crystal lattice represented in a VASP POSCAR file. Supported strain types include uniaxial, biaxial, shear, and hydrostatic. It is useful for first-principles simulations of strained materials.

\section{Installation Requirements}
\begin{itemize}
    \item Python 3.8 or higher
    \item ASE (\texttt{pip install ase})
    \item NumPy (\texttt{pip install numpy})
\end{itemize}

\section{Strain Types Supported}

\begin{tabular}{|l|l|l|}
\hline
\textbf{Type} & \textbf{Direction(s)} & \textbf{Description} \\
\hline
Uniaxial & x, y, z & Stretch/compress along one axis \\
Shear & xy, yz, xz & Distort lattice shape without uniform scaling \\
Biaxial & xy-biaxial, yz-biaxial, xz-biaxial & Expand/compress two axes simultaneously \\
Hydrostatic & xyz & Uniform expansion/compression along all axes \\
\hline
\end{tabular}

\section{Mathematical Formalism}

\subsection{Strain Tensor}
Strain is represented by a second-order tensor:
\[
\boldsymbol{\epsilon} =
\begin{pmatrix}
\epsilon_{xx} & \epsilon_{xy} & \epsilon_{xz} \\
\epsilon_{yx} & \epsilon_{yy} & \epsilon_{yz} \\
\epsilon_{zx} & \epsilon_{zy} & \epsilon_{zz}
\end{pmatrix}
\]

\paragraph{Uniaxial strain:} Only the diagonal element along the strained axis is non-zero. Example, x-direction:
\[
\epsilon_{xx} = \text{strain\_value}, \quad
\epsilon_{yy} = 0, \quad
\epsilon_{zz} = 0
\]

\paragraph{Biaxial strain:} Two diagonal elements non-zero. Example, xy-plane:
\[
\epsilon_{xx} = \epsilon_{yy} = \text{strain\_value}, \quad
\epsilon_{zz} = 0
\]

\paragraph{Shear strain:} Off-diagonal elements non-zero. Example, xy-plane:
\[
\epsilon_{xy} = \epsilon_{yx} = \text{strain\_value}
\]

\paragraph{Hydrostatic strain:} All diagonal elements equal:
\[
\epsilon_{xx} = \epsilon_{yy} = \epsilon_{zz} = \text{strain\_value}
\]

\subsection{Deformation Matrix}
The deformation gradient matrix $\mathbf{F}$ is:
\[
\mathbf{F} = \mathbf{I} + \boldsymbol{\epsilon}
\]
where $\mathbf{I}$ is the $3 \times 3$ identity matrix.

\subsection{Applying Strain to the Lattice}
The new lattice vectors $\mathbf{a}'_i$ are computed as:
\[
\mathbf{a}'_i = \mathbf{F} \cdot \mathbf{a}_i
\]
where $\mathbf{a}_i$ are the original lattice vectors.

Atomic positions are scaled automatically:
\[
\mathbf{r}'_j = \mathbf{F} \cdot \mathbf{r}_j
\]
where $\mathbf{r}_j$ are original atomic positions and $\mathbf{r}'_j$ are strained positions.

\section{Script Functions}
\begin{itemize}
    \item \textbf{apply\_strain(atoms, strain\_direction, strain\_percentage)}: Applies strain tensor to ASE Atoms object.
    \item \textbf{read\_poscar(poscar\_file)}: Reads POSCAR file into ASE Atoms object.
    \item \textbf{write\_poscar(atoms, output\_file)}: Writes ASE Atoms to POSCAR.
    \item \textbf{main()}: Command-line interface for batch usage.
\end{itemize}

\section{Usage Examples}
\begin{verbatim}
# Uniaxial strain along x-axis by 2%
python apply_strain.py -i POSCAR -d x -s 2.0 -o POSCAR_strained

# Biaxial strain in xy-plane by -1.5%
python apply_strain.py -i POSCAR -d xy-biaxial -s -1.5 -o POSCAR_biaxial

# Hydrostatic strain (uniform expansion) by 0.5%
python apply_strain.py -i POSCAR -d xyz -s 0.5 -o POSCAR_hydro
\end{verbatim}

\section{Notes and Best Practices}
\begin{itemize}
    \item ASE uses double precision for atomic positions.
    \item For fixed-cell relaxation in VASP, set \texttt{ISIF=2}.
    \item Positive strain: expansion; negative strain: compression.
    \item Invalid strain directions raise \texttt{ValueError}.
\end{itemize}

\section{Optional Improvements}
\begin{itemize}
    \item Loop over multiple strain percentages.
    \item Logging and automated file naming.
    \item Integration with high-throughput DFT pipelines.
\end{itemize}

\section{References}
\begin{itemize}
    \item ASE Documentation: \url{https://wiki.fysik.dtu.dk/ase/}
    \item VASP POSCAR Format: \url{https://www.vasp.at/}
\end{itemize}

\end{document}
